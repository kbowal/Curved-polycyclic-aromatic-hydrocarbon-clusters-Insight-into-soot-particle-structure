\newcommand{\curPAHAP}{curPAHIP\xspace} 
%
\section{Introduction}
\label{sec:Introduction}
% 
%% Curved carbons are found in porous carbons, glassy carbons, soot, etc


%% Curvature is caused by non-hexagonal rings, causes special steric and electronic properties


%% Curved carbons have great promise in many applications such as understanding particulates, novel nanocarbon materials, batteries, drug delivery, sensors, gas storage, etc
Preliminary simulations show that cluster arrangment of curved corannulene is different than planar coronene (Preprint 214). This may be more representative of the experimentally observed structure of nascent soot particles.

Further areas of interest include: arrangement of curved and planar PAH clusters, structure of cPAH clusters containing ion(s)

The structural properties of cPAH clusters are unknown and further molecular modelling studies are required to provide insight into the clustering behaviour of cPAHs - their size-dependent arrangements, interactions with planar PAHs, and assembly around ions.

Understanding non-covalent interactions with and between curved carbon nanostructures has importance in many systems and great potential for numerous applications.

Motivation: structure of soot particles / interstellar medium, applications such as electronics, nanomedicine, energy, etc



%% Previous work looking at cPAH interactions:	Homogeneous dimers
Significant energy between nested concave-to-convex cPAHs so curvature doesn’t prevent dimers from forming pi-pi stacked assemblies similar to planar systems (10.1002/qua.21794). Curvature increases interaction strength by decreasing C-C distances for increased dispersion interactions (10.1021/jp305700k). Very curved PAHs may have interaction strengths similar to fPAHs though because of increased sterics (10.1016/j.proci.2018.05.046, 10.1002/qua.21794?

As with fPAHs, cPAH dimer interactions are dominated by $\pi$-$\pi$ interactions compared to the CH-$\pi$ interactions. However cPAHs show different behaviour than fPAHs, with eclipsed dimer showing more stability than the staggered dimer (10.1016/j.cplett.2011.07.030).



%% Previous work looking at cPAH interactions:	Bulk systems
Corannulene has no long-range stacked order in crystal analysis  - CH-pi interactions dominate (10.1107/S0567740876012430)
cPAHs with increased curvature and rigidity and heteroatoms form columnar stacks (many refs, including 10.1021/cg100898g)
Condensed / larger systems: pi-H stacking in crystal (10.1515/znb-2010-0403) 

So expect homogeneous corannulene nanocluster to be disordered, homo 2pent15ring cluster to have stacks

Mixed cluster – expect to be dominated by 2pent15rings but enhanced corannulene stacking.


%% Previous work looking at cPAH interactions:	Heterogeneous systems (dimers, etc)
Planar PAH studies show that heterogeneity decreases the stability of the nanocluster since heterogeneous dimers are significantly weaker.  This leads to a distinct partitioning 

something about cPAH hetero dimers/systems...

Thus we hypothesise that heterogeneous cPAH clusters may be more stable than their heterogeneous fPAH counterparts since different sizes of cPAHs may provide stable packing arrangements.


%% Limitations: mostly static DFT calculations, no information about nanoparticles. 



%% Following previous work looking at heterogeneous fPAHs…
Previous work examined heterogeneous PAH clusters using replica exchange molecular dynamics (ref)...

%% 
The purpose of this work is to explore properties (molecular arrangements, melting points, densities, surface properties, etc) of clusters containing curved PAHs.  
Hypothesis: cPAH clusters are more representative of the experimentally observed structure of nascent soot particles, heterogeneous cPAH clusters are significant since they are stable than their fPAH counterparts.
This is important for better understanding of soot particle formation and growth, and has implications for other carbon nanomaterials such as ???

%
\section{Conclusions}


\section*{Acknowledgements}
This work used the ARCHER UK National Supercomputing Service (\url{http://www.archer.ac.uk}).
K.B. is grateful to the Cambridge Trust and the Stanley Studentship at King's College, Cambridge for their financial support.
This project is also supported by the National Research Foundation (NRF), Prime Minister's Office, Singapore under its Campus for Research Excellence and Technological Enterprise (CREATE) programme.
