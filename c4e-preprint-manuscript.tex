\newcommand{\curPAHAP}{curPAHIP\xspace} 
%
\section{Introduction}
\label{sec:Introduction}
% Answering: What is the energy/structure of cPAH particles? How does this differ from fPAH particles? What is the influence of molecule size, proportion, and presence of ions or fPAHs?

%% Curved carbons are everywhere...
Curved carbons are found in porous carbons, glassy carbons, soot, etc
% even important in small proportions since seem to interrupt formation of mesophase...
% Mesophase pitch is used to produce many materials, including synthetic graphite.

It has been suggested that the presence of curved PAH molecules prevents graphitisation, since curvature diminishes the anisotropy which provides reduced connectivity in lower (1 and 2) dimensions. In other words, curvature disrupts formation of the mesophase.

We know that the integration of curvature in molecules is crucial in determining whether a combusted wood-based material produces graphitic or non-graphitised carbon (not oxygen as also hypothesised) (Abrahamson, Joseph P., et al. "Carbon structure and the resulting graphitizability upon oxygen evolution." Carbon 135 (2018): 171-179).

Curved aromatics seem to form fullerenes and then ribbons and net-negative structures (Abrahamson, Joseph, et al. "Trajectories of graphitizable anthracene coke and non-graphitizable sucrose char during the earliest stages of annealing by rapid CO2 laser heating." C 4.2 (2018): 36; Jake's papers).


% Curvature is caused by... (incl experimental refs)
Curvature is caused by the presence of non-hexagonal rings within a hexagonal lattice. This causes special steric and electronic properties...

% Arrangement of atoms determines everything...
The arrangement of atoms in a carbon material determines its structure and properties and function. This nanostructure is crucial to understand the existing and potential behaviour of carbon materials. For example, to understand (how to optimise) structure's stability/longevity, environmental interactions, ability to store other molecules -- all important for gas storage or sequestration applications.  Similar questions about the detailed structure of the material are asked for other applications such as batteries, adsorbents, involving thermal protection, conductivity.
Need to know this in order to optimise the nanostructures for these applications.

%% curved carbon applications...
Curved carbons have great promise in many applications such as understanding particulates, novel nanocarbon materials, batteries, drug delivery, sensors, gas storage, etc

Applications: Lithium ion batteries, electronics, fuel cells

% from Carbon website: "Relevant application areas for carbon materials include, but are not limited to: biology and medicine; catalysis; electronic, optoelectronic, spintronic, high-frequency, and photonic devices; energy storage and conversion systems; environmental applications and water treatment; smart materials and systems; and, structural and thermal applications."


%% Previous work looking at cPAH interactions:	Homogeneous dimers
Significant energy between nested concave-to-convex cPAHs so curvature doesn't prevent dimers from forming pi-pi stacked assemblies similar to planar systems (10.1002/qua.21794, 10.1002/jcc.25084). Curvature increases interaction strength by decreasing C-C distances for increased dispersion interactions (10.1021/jp305700k). Very curved PAHs may have interaction strengths similar to fPAHs though because of increased sterics (10.1016/j.proci.2018.05.046, 10.1002/qua.21794?

As with fPAHs, cPAH dimer interactions are dominated by $\pi$-$\pi$ interactions compared to the CH-$\pi$ interactions. However cPAHs in the concave-convex dimer configuration show different behaviour than fPAH dimesr and cPAHs in convex-convex configuration, with eclipsed dimer showing more stability than the staggered dimer (10.1016/j.cplett.2011.07.030, 10.1002/jcc.25084).
Dispersion is dominant energy contribution, electrostatics are also significant in contrast to fPAH dimers (10.1002/jcc.25084, 10.1016/j.cplett.2011.07.030). Pi-pi > pi-CH interactions, eclipsed > staggered.

Different degrees of curvature can result in increased or decreased cPAH dimer strengths, due to geometry and (less so) electrostatic effects. Increased curvature can increase interaction strength by decreasing C-C distances for increased dispersion interactions (10.1021/jp305700k). Very curved PAHs may have interaction strengths similar to fPAHs though because of increased sterics (10.1016/j.proci.2018.05.046, 10.1002/qua.21794). High curvature increases exchange-repulsion and can destabilise dimer (10.1021/jp305700k, 10.1002/qua.21794).

Homodimers are weaker than heterodimers because they don't take advantage of bowl complementarity and additional heteroatom interactions that provide further electrostatic stabilisation (10.1002/jcc.25084).


%% Previous work looking at cPAH interactions:	Bulk systems
Corannulene has no long-range stacked order in crystal analysis (10.1107/S0567740876012430, 10.1021/jo050233e, 10.1021/acs.analchem.8b05260, 10.1016/j.carbon.2015.06.041, 10.1021/ar950197d). CH-pi interactions dominate. Shallow bowl, rapid bowl inversion.
CH-$\pi$ stacking, where the posivitely charged rim of one molecule is almost perpendicular to the negative region and the bottom of the other molecule in a T-shaped configuration, exists in the corannulene crystal (10.1515/znb-2010-0403). This is supported by evaluation of the electrostatic potential of corannulene.

cPAHs with increased size, curvature, rigidity, and/or heteroatoms form columnar stacks (10.1021/ar950197d, 10.1039/C39940002571, 10.1021/ja0123148, 10.1021/ja970845j, 10.1021/acs.jpcc.6b10895, 10.1016/j.carbon.2015.06.041, 10.1002/anie.200460855, 10.1021/om040131z, 10.1021/cg100898g, 10.1021/ja0518169, 10.1039/A905860E, 10.1515/znb-2010-0403, 10.1021/cr050554q). Larger pi-pi overlap. Staggered configs allow extended pi network enhanced by CH-pi interactions.

cPAH interactions with different sized molecules (no heteroatoms) (refs????). Increased stability compared to heterogeneous fPAH systems because of shape fitting and enhanced electrostatics and CH-pi interactions (?).

So expect homogeneous corannulene nanocluster to be disordered, homo 2pent15ring cluster to have stacks. Expect mixed cPAH cluster to be dominated by 2pent15rings but enhanced corannulene stacking and more stable than comparable mixed fPAH cluster.


%% Previous work looking at cPAH interactions:	Heterogeneous systems (dimers, etc)
Planar PAH studies show that heterogeneity decreases the stability of the nanocluster since heterogeneous dimers are significantly weaker.  This leads to a distinct partitioning...

something about cPAH hetero dimers/systems...
Homodimers are weaker than heterodimers because they don't take advantage of bowl complementarity and additional heteroatom interactions that provide further electrostatic stabilisation (10.1002/jcc.25084).

Thus we hypothesise that heterogeneous cPAH clusters may be more stable than their heterogeneous fPAH counterparts since different sizes of cPAHs may provide stable packing arrangements.

%% previous work on cPAH nucleation
Preliminary simulations show that cluster arrangment of curved corannulene is different than planar coronene (Preprint 214). This may be more representative of the experimentally observed structure of nascent soot particles.


Further areas of interest include: arrangement of curved and planar PAH clusters, structure of cPAH clusters containing ion(s)

The structural properties of cPAH clusters are unknown and further molecular modelling studies are required to provide insight into the clustering behaviour of cPAHs - their size-dependent arrangements, interactions with planar PAHs, and assembly around ions.

Understanding non-covalent interactions with and between curved carbon nanostructures has importance in many systems and great potential for numerous applications.

Motivation: structure of soot particles / interstellar medium, applications such as electronics, nanomedicine, energy, etc


%% Previous work looking at heterogeneous fPAHs…
Previous work examined heterogeneous PAH clusters using replica exchange molecular dynamics (ref)...

%% Limitations of existing work
To date, detailed studies of cPAHs have primarily included static DFT calculations or crystal structure experiments, neither of which provide information about intermolecular dynamics and particle nanostructure.


We should note that in this work, the terms (nano)particle and cluster are effectively synonymous: a nanoparticle is made up of a cluster of molecules.


%% purpose of this work... 
The purpose of this work is to explore properties (molecular arrangements, melting points, densities, surface properties, etc) of clusters containing curved PAHs.  

In this work, we are motivated to understand the influence of cPAHs in soot particle structure and stability, but these results have general relevance to nanoparticles containing cPAHs regardless of source or application.

Hypothesis: cPAH clusters are more representative of the experimentally observed structure of nascent soot particles, heterogeneous cPAH clusters are significant since they are stable than their fPAH counterparts.

This is important for better understanding of soot particle formation and growth, and has implications for other carbon nanomaterials such as batteries, imaging probes, gas storage, optoelectronics, and targeted nanomedicine.

%in this work...
We will address these questions in this work by studying nanoparticle systems containing cPAHs, first in homogeneous clusters containing one molecule type only, then extending this to heterogeneous clusters to understand the interactions and effect of cPAH size and ratio. Finally, exploratory work on systems containing cation(s) and fPAHs is presented.

%question: do curved PAHs partition within clusters the same way fPAHs do?
%question: Are curved PAH clusters more stable than fPAH clusters? Predict long-range dipole-dipole interactions increase stability.


%%%%%%%%%%%%%% Other notes to perhaps include %%%%%%%%%%%%%
Pi-pi interactions play a pivotal role in so many super important processes (the way proteins fold and drugs bind to targets, crystal packing and charge transport properties in organic electronics, materials) - see references “Pi-pi Stacking of Curved Carbon Networks: The Corannulene Dimer” in Mendeley

Relevance to carbon (quantum) dots?




\section{Methods}
%% systems
Clusters studied: homogeneous (corannulene, 2pent15ring each: 25, 40, 50, 100, [200]), heterogeneous (20/20, 10/30, 30/10).
40 corannulene with 1 K+ and 2 K+.

%% curPAHIP
%Reintroduce curPAHIP potential
We have previously developed an atomic potential for cPAHs, based on the intermolecular potential for planar PAHs 



%% Computational details
Replica exchange molecular dynamics, position potential (Gromacs 5.1.4, mixed initial configurations using packmol, 3 ns, 60-75 replicas, 200 - 1600 K).
Density functional theory.

%% Analyses
%Melting point analysis.
The REMD method allows rapid evaluation of equilibrated systems, but the replica swapping process hides local ensemble information such as a change in energy due to melting. This means that information about the cluster melting points could not be extracted from the REMD simulations. To assess the melting points of these systems, individual 1 ns simulations using classical MD were conducted at each desired temperature from the final REMD configuration. No position potentials were implemented for these post-REMD MD simulations.
The final 500 ps of the post-REMD MD simulation was used as the production period and is used for analysis.
(The final 500 ps of the post-REMD MD simulations are equilibrated and show <5\% drift in energy so this was selected as the production period.)

Melting points are identified using both energy- and movement-based metrics, as in previous works (many refs).
Previous experimental and simulation-based studies have shown that a LI value of 0.10-0.15 provides an indication of solid-to-liquid transition temperature (10.1063/1.1426419).

Molecular arrangement analysis (radial distances, coordination numbers, stacking angle, etc).

A quantitative measure of the degree of stacking order in the molecular structure is provided through the use of coordination numbers. These are calculated as the number of near neighbours within a cut-off radius $R$ of each molecule, averaged over each molecule type. The values of $R$ were selected to include sandwich-type stacked interactions between molecules but exclude molecules more than one layer away. (This means that the equilibrium dimer distances served as the minimum $R$ values for each molecule type). For corannulene molecules this cut-off distance, $R_{\text{ANN}}$, is 0.4 nm, and for 2pent15ring molecules $R_{\text{two}}$ is 0.5 nm.  The sensitivity of the coordination number values on these selected $R$ is discussed further in the Supplementary Information.


\section{Results}
\subsection{curPAHIP potential for larger cPAHs - dimer energies}
The curPAHIP potential developed from corannulene energies provides good results for the larger 2pent15ring molecule (within 5\% of the dispersion-corrected DFT). In contrast, the isoPAHAP gives 31\% smaller minimum energy than B97D.


\subsection{Energies}
The intermolecular energies as a function of cluster mass are shown in Figure XXX (tabulated values are provided in the Supplementary Information).  The energies are shown per atom in the system, in order to facilitate direct comparison between systems of different molecule sizes.
It is clearly seen that in all cases the energy decreases with cluster mass, for clusters containing curved or planar PAHs.  Homogeneous clusters containing planar PAHs show relatively consistent energy trends across molecule sizes from pyrene to circumcoronene (all coloured the same here to allow for ease in reading), with heterogeneous clusters generally at lower energy values.  Clusters containing the cPAH 2pent15ring show energies similar to those of heterogeneous fPAH clusters, while clusters containing the cPAH corannulene have lower energies.
The effect of heterogeneity in cPAH clusters shows a distinct effect of molecular ratio: an increased proportion of 2pent15ring in the cluster decreases the energy. At 50\%/50\% molecule type partitioning, the heterogeneous cluster reflects that of a homogeneous corannulene cluster (across cluster size). However, when the cluster contains only 25\% 2pent15ring, the cluster has significantly higher energy and inversely with 75\% 2pent15ring, the cluster's energy decreases significantly.  This suggests that the presence of 2pent15ring molecules increases cluster stability if above 50\% composition.
The presence of cation(s) in the cluster had opposite effects on clusters containing the two molecule types: 2pent15ring clusters with K+ showed an increased energy, while clusters with corannulene molecules containing K+ showed a decreased energy in proportion with the number of ions.  These differences are shown by black arrows in Figure XXX. This suggests that the ion influences molecular interactions and cluster structure in different ways depending on the molecule size.
Finally, the energy of a cluster containing both curved and planar PAHs has an energy similar to that of homogeneous coronene clusters, showing that in these cases (not including some advanced interactions such as induced dipoles, 50/50\% split, very similar sizes so less cooperative perhaps) the mixing of molecule types does not enhance the interaction energy - it remains similar to that of the weaker component.  Further work to thoroughly develop/assess a fPAH-cPAH intermolecular potential and test the effect of molecular composition should be done to further explore the possible synergistic effects of clusters containing cPAHs and fPAHs.

%include figure of intermolecular energy per atom vs cluster mass (with fPAH, cPAH; homo and hetero). Lines shown for 2pent15ring and ann clusters to guide the eye.



\subsection{Homogeneous clusters}
\subsubsection{Structure}
Tightly packed clusters (densities?)
Significant stacking of 2pent15ring molecules but not much for corannulene (RDFs, cluster snapshots, tilting angles).
% 10.1021/cg100898g crucial paper to compare with stacking of 2pent15ring molecule!
% Lots of detail on molecule spacings, slipping angles, motifs etc of cPAH clusters in 10.1002/chem.201303357 -- definitely read/incorporate in cPAH cluster structure work!



\subsubsection{Melting points}
Bulk and cluster melting points of corannulene and 2pent15rings systems.
Compare directly to melting points of fPAH clusters (Dongping).


\subsection{Heterogeneous clusters}
Same core-shell partitioning as planar PAHs (big in centre, small in shell) but not as distinct - effect of molecule sizes. (radial distances, snapshots, coordination numbers, tilting angles, histograms)
Effect of ratio...
Effect of temperature...


\subsection{Coordination number}
Fig XX shows results from all types of clusters, each containing 40 PAH molecules.  The coordination number for the molecule types are shown.  Further analysis was conducted to determine which molecule types were present as near neighbours within heterogeneous systems.  We saw that all corannulene stacking interactions came from neighbouring 2pent15ring molecules.  The 2pent15ring molecule stacking interactions came predominantly from other 2pent15ring molecules, however a proportion of the interactions (shown as insets with horizontal lines in the 2pent15ring columns of Fig XX) are with corannulene molecules.

Include bar chart showing CN values for all clusters containing 40 molecules.


%Further applications: Understanding the stacking behaviour of curved PAHs is relevant to materials studies / supramolecular chemistry / crystalline organic materials because of their unique optical and conduction properties; and can also lead to using these molecules as building blocks for engineering organic crystals with desired properties.  Synthesis of curved PAH hybrids has shown that the $\pi-\pi$ interactions of these curved species can result in tight stacking in 1D columns (10.1021/acs.cgd.7b01258).
% relevant to polymers of intrinsic microporosity, graphene systems, graphene quantum dots (experimental condensation structure)


\section{Discussion}
Compare with experimental morphologies (do we see sections of curved/planar together?)

Future work: cation fixed in centre, cPAH and fPAH mixed clusters

\section{Conclusions}


\section*{Acknowledgements}
This work used the ARCHER UK National Supercomputing Service (\url{http://www.archer.ac.uk}).
K.B. is grateful to the Cambridge Trust and the Stanley Studentship at King's College, Cambridge for their financial support.
This project is also supported by the National Research Foundation (NRF), Prime Minister's Office, Singapore under its Campus for Research Excellence and Technological Enterprise (CREATE) programme.
