\newcommand{\curPAHAP}{curPAHIP\xspace} 
%
\section{Introduction}
\label{sec:Introduction}
% 
%% Curved carbons are found in porous carbons, glassy carbons, soot, etc


%% Curvature is caused by non-hexagonal rings, causes special steric and electronic properties


%% Curved carbons have great promise in many applications such as understanding particulates, novel nanocarbon materials, batteries, drug delivery, sensors, gas storage, etc
Preliminary simulations show that cluster arrangment of curved corannulene is different than planar coronene (Preprint 214). This may be more representative of the experimentally observed structure of nascent soot particles.

Further areas of interest include: arrangement of curved and planar PAH clusters, structure of cPAH clusters containing ion(s)

The structural properties of cPAH clusters are unknown and further molecular modelling studies are required to provide insight into the clustering behaviour of cPAHs - their size-dependent arrangements, interactions with planar PAHs, and assembly around ions.

Understanding non-covalent interactions with and between curved carbon nanostructures has importance in many systems and great potential for numerous applications.

Motivation: structure of soot particles / interstellar medium, applications such as electronics, nanomedicine, energy, etc



%% Previous work looking at cPAH interactions:	Homogeneous dimers
Significant energy between nested concave-to-convex cPAHs so curvature doesn't prevent dimers from forming pi-pi stacked assemblies similar to planar systems (10.1002/qua.21794). Curvature increases interaction strength by decreasing C-C distances for increased dispersion interactions (10.1021/jp305700k). Very curved PAHs may have interaction strengths similar to fPAHs though because of increased sterics (10.1016/j.proci.2018.05.046, 10.1002/qua.21794?

As with fPAHs, cPAH dimer interactions are dominated by $\pi$-$\pi$ interactions compared to the CH-$\pi$ interactions. However cPAHs in the concave-convex dimer configuration show different behaviour than fPAH dimesr and cPAHs in convex-convex configuration, with eclipsed dimer showing more stability than the staggered dimer (10.1016/j.cplett.2011.07.030, 10.1002/jcc.25084).

Significant energy between nested concave-convex cPAHs so curvature doesn’t prevent dimers from forming pi-pi stacked assemblies similar to planar systems (10.1002/qua.21794, 10.1002/jcc.25084).
Dispersion is dominant energy contribution, electrostatics are also significant in contrast to fPAH dimers (10.1002/jcc.25084, 10.1016/j.cplett.2011.07.030). Pi-pi > pi-CH interactions, eclipsed > staggered.

Different degrees of curvature can result in increased or decreased cPAH dimer strengths, due to geometry and (less so) electrostatic effects. Increased curvature can increase interaction strength by decreasing C-C distances for increased dispersion interactions (10.1021/jp305700k). Very curved PAHs may have interaction strengths similar to fPAHs though because of increased sterics (10.1016/j.proci.2018.05.046, 10.1002/qua.21794). High curvature increases exchange-repulsion and can destabilise dimer (10.1021/jp305700k, 10.1002/qua.21794).

Homodimers are weaker than heterodimers because they don't take advantage of bowl complementarity and additional heteroatom interactions that provide further electrostatic stabilisation (10.1002/jcc.25084).


%% Previous work looking at cPAH interactions:	Bulk systems
Corannulene has no long-range stacked order in crystal analysis (10.1107/S0567740876012430, 10.1021/jo050233e, 10.1021/acs.analchem.8b05260, 10.1016/j.carbon.2015.06.041, 10.1021/ar950197d). CH-pi interactions dominate. Shallow bowl, rapid bowl inversion.
CH-$\pi$ stacking, where the posivitely charged rim of one molecule is almost perpendicular to the negative region and the bottom of the other molecule in a T-shaped configuration, exists in the corannulene crystal (10.1515/znb-2010-0403). This is supported by evaluation of the electrostatic potential of corannulene.

cPAHs with increased size, curvature, rigidity, and/or heteroatoms form columnar stacks (10.1021/ar950197d, 10.1039/C39940002571, 10.1021/ja0123148, 10.1021/ja970845j, 10.1021/acs.jpcc.6b10895, 10.1016/j.carbon.2015.06.041, 10.1002/anie.200460855, 10.1021/om040131z, 10.1021/cg100898g, 10.1021/ja0518169, 10.1039/A905860E, 10.1515/znb-2010-0403, 10.1021/cr050554q). Larger pi-pi overlap. Staggered configs allow extended pi network enhanced by CH-pi interactions.


cPAH interactions with different sized molecules (no heteroatoms) (refs????). Increased stability compared to heterogeneous fPAH systems because of shape fitting and enhanced electrostatics and CH-pi interactions (?).

So expect homogeneous corannulene nanocluster to be disordered, homo 2pent15ring cluster to have stacks. Expect mixed cPAH cluster to be dominated by 2pent15rings but enhanced corannulene stacking and more stable than comparable mixed fPAH cluster.


%% Previous work looking at cPAH interactions:	Heterogeneous systems (dimers, etc)
Planar PAH studies show that heterogeneity decreases the stability of the nanocluster since heterogeneous dimers are significantly weaker.  This leads to a distinct partitioning...

something about cPAH hetero dimers/systems...
Homodimers are weaker than heterodimers because they don't take advantage of bowl complementarity and additional heteroatom interactions that provide further electrostatic stabilisation (10.1002/jcc.25084).

Thus we hypothesise that heterogeneous cPAH clusters may be more stable than their heterogeneous fPAH counterparts since different sizes of cPAHs may provide stable packing arrangements.


%% Limitations: mostly static DFT calculations, no information about nanoparticles. 



%% Following previous work looking at heterogeneous fPAHs…
Previous work examined heterogeneous PAH clusters using replica exchange molecular dynamics (ref)...

%% 
The purpose of this work is to explore properties (molecular arrangements, melting points, densities, surface properties, etc) of clusters containing curved PAHs.  

Hypothesis: cPAH clusters are more representative of the experimentally observed structure of nascent soot particles, heterogeneous cPAH clusters are significant since they are stable than their fPAH counterparts.

This is important for better understanding of soot particle formation and growth, and has implications for other carbon nanomaterials such as batteries, imaging probes, gas storage, optoelectronics, and targeted nanomedicine.

\section{Methods}
%% systems
Clusters studied: homogeneous (100 corannulene, 100 2pent15ring), heterogeneous (20/20, 10/30, 30/10)

%% curPAHIP
Reintroduce curPAHIP potential

%% Computational details
Replica exchange molecular dynamics, position potential (Gromacs 5.1.4, mixed initial configurations using packmol, 3 ns, 60-75 replicas, 200 - 1600 K).
Density functional theory.

%% Analyses
Melting point analysis.
Molecular arrangement analysis (radial distances, coordination numbers, stacking angle, etc).


\section{Results}
\subsection{curPAHIP potential for larger cPAHs - dimer energies}
The curPAHIP potential developed from corannulene energies provides good results for the larger 2pent15ring molecule (within 5\% of the dispersion-corrected DFT). In contrast, the isoPAHAP gives 31\% smaller minimum energy than B97D.

\subsection{Homogeneous clusters}
\subsubsection{Structure}
Tightly packed clusters (densities?)
Significant stacking of 2pent15ring molecules but not much for corannulene (RDFs, cluster snapshots, tilting angles).
%10.1021/cg100898g crucial paper to compare with stacking of 2pent15ring molecule!
%Lots of detail on molecule spacings, slipping angles, motifs etc of cPAH clusters in 10.1002/chem.201303357 -- definitely read/incorporate in cPAH cluster structure work!

\subsubsection{Melting points}
Bulk and cluster melting points of corannulene and 2pent15rings systems.
Compare directly to melting points of fPAH clusters (Dongping).


\subsection{Heterogeneous clusters}
Same core-shell partitioning as planar PAHs (big in centre, small in shell) but not as distinct - effect of molecule sizes. (radial distances, snapshots, coordination numbers, tilting angles, histograms)
Effect of ratio...
Effect of temperature...



\section{Discussion}
Compare with experimental morphologies (do we see sections of curved/planar together?)

Future work: cation fixed in centre, cPAH and fPAH mixed clusters

\section{Conclusions}


\section*{Acknowledgements}
This work used the ARCHER UK National Supercomputing Service (\url{http://www.archer.ac.uk}).
K.B. is grateful to the Cambridge Trust and the Stanley Studentship at King's College, Cambridge for their financial support.
This project is also supported by the National Research Foundation (NRF), Prime Minister's Office, Singapore under its Campus for Research Excellence and Technological Enterprise (CREATE) programme.
